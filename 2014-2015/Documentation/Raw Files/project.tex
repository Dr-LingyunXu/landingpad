% !TEX root = UAV_Landing_Pad.tex


\chapter{Project Overview}
This section contains information regarding the team, project progress, and our testing methodology


\section{Team Members and Roles}
The team consists of the following members.

\begin{itemize}

  \item Colter Assman - UAV Specialist.  Colter is a member of SDSM\&T's UAV team and liaison for Search and Rescue, FAA, and ourselves.
  \item Julian Brackins - Scrum Master.  Julian is responsible for documentation, team management, presentation preparation, and craft orientation software for UAV landing.
  \item Samuel Carroll - Navigations Specialist focused in developing object avoidance subroutines and UGV construction.
  \item Hafiza Farzami - Control Panel designer. Also in charge of AR Tag implementation for determining craft pose and positioning for UAV landing.
  \item Charles Parsons - Navigations Specialist.  Charles has experience with various SLAM algorithms and their integrations and is investigating the path planning procedures for the UGV.
  \item Alex Wulff - Technical Lead.  Alex resolves technical issues and is responsible for overseeing informal code reviews. Alex has been the architect for the custom ROS API layer, SVN logger for monitoring team collaboration, and the lead designer for the UGV construction.

\end{itemize}


\section{Project  Management Approach}
The structure of Eye in the Sky's development is based on Agile methodology.  There are a series of six formal three-week sprints through out the development of this solution with a short sprint during the SDSM\&T winter break (Sprinter Break).  All backlog structure is housed on Trello, an internet resource for project management.  

The development process will focus heavily on code review and simulation testing before builds are pushed to the hardware.  This is primarily due to the budget constraints on this project, and we can not afford continual repairs on either the UGV or UAV.  All \hyperlink{testing}{System and Unit Testing} is available here.


\section{Phase  Overview}

The initial sprints for this project have been primarily research driven.  As a result Sprints 4,5, and 6 will focus on development of UAV / UGV in parallel.  For development, the following \hyperlink{testing}{System and Unit Testing} structure has been put into place.

\section{Terminology and Acronyms}
\begin{itemize}
%At this point I'll be adding this willy nilly but we'll alphabetize them at some point.
	\item UAV - Unmanned Aerial Vehicle. Commonly referred to as a drone.
	\item UGV - Unmanned Ground Vehicle
	\item Point cloud - A set of data points in a given coordinate system.
	\item QR Code - Quick Response Code. A matrix barcode first used for the automotive industry in Japan. Has recently become popular due to quick readability and increased information capacity over standard UPC barcodes
	\item AR Tag - A marker used to support augmented reality.
\end{itemize}
%Provide a list of terms used in the document that warrant definition.  Consider 
%industry or domain specific terms and acronyms as well as system specific. 
