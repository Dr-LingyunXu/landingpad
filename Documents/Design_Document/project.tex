% !TEX root = SystemTemplate.tex


\chapter{Project Overview}
This section provides an overview of project management and a brief overview of the phases of the project.



\section{Team Members and Roles}
Team members and assigned roles:
\begin{itemize}
\item \textbf{Steven Huerta} - Team Lead, Simulation, AI Landing
\item \textbf{Christopher Smith} - Simulation, AI Landing
\item \textbf{Jonathan Dixon} - Visual Homography Landing
\item \textbf{Julian Brackins} - Visual Homography Landing
\item \textbf{Dylan Geyer} - UAV Build
\end{itemize}

\noindent Role assignments:
\begin{itemize}
\item \textbf{UAV Build} - Responsible for the physical assembly of the hexrotor, including wiring and mounting of hardware on the UAV.
\item \textbf{Visual Homography Landing} - Responsible for the development of an algorithm to calculate the distance and direction to the center of the landing pad. 
\item \textbf{AI Landing} - Responsible for the development of a landing algorithm utilizing an AI approach such as a neural net.
\item \textbf{Simulation} - Responsible for the development of a test environment wherein a simulated hexrotor utilizes the developed landing algorithms to evaluate the fitness of the landing algorithm.  
\end{itemize}



\section{Project  Management Approach}
This project is managed through the Agile framework. The sprints are three weeks with a one week pause between for evaluation and plan adaptation before the next sprint. Sprints 1, 2, and 3 comprise Phase 1. Sprints 4, 5, and 6 comprise Phase 2. Requirements for the project were collected by interviewing the client to form a set of user stories. These user stories provide a backlog managed on Trello. The user stories for this project represent composition of several tasks. Tasks are factored from the user stories and assigned to team members as that task relates to the project. These task assignments and their progress are managed on Trello. Issues and troubleshooting is handled through email and team meetings. The team division of work is to limit the dependencies between team members, so that issues encountered by one team member does not impact the progress of other team members. Tasks are sometimes added, altered, or removed so as to reflect a better size-up of the backlog item.



%This section will provide an explanation of the basic approach to managing the 
%project.  Typically, this would detail how the project will be managed through 
%a given Agile methodology.  The sprint length (i.e. 2 weeks) and product backlog 
%ownership and location (ex. Trello) are examples of what will be discussed.  An 
%overview of the system used to track sprint tasks, bug or trouble tickets, and 
%user stories would be warranted. 


\section{Phase  Overview}
\normalsize{\textbf{Phase 1}}
\begin{itemize}
\item \textbf{O-1}: As an owner, I want the UAV to autonomously take-off from the landing pad.
\item \textbf{O-2}: As an owner, I want the UAV to autonomously navigate through a series of waypoints.
\end{itemize}

\vspace{3mm}
\noindent\normalsize{\textbf{Phase 2}}
\begin{itemize}
\item \textbf{U-1}: As a user, I want to communicate the waypoints to the UAV.
\item \textbf{O-3}: As an owner, I want the UAV to autonomously return to the location of the landing pad.
\item \textbf{O-4}: As an owner, I want the UAV to autonomously land on the landing pad without damaging the craft.
\item \textbf{O-5}: As an owner, I want the UAV to autonomously land on the landing pad with the correct orientation.
\end{itemize}

%If the system will be implemented in phases, describe those phases/sub-phases (design, 
%implementation, testing, delivery) and the various milestones in this section. 
% This section should also contain a correlation between the phases of development 
%and the associated versioning of the system, i.e. major version, minor version, 
%revision. 

\section{Terminology and Acronyms}
\begin{itemize}
\item FCU - Flight Control Unit
\item ROS - Robot Operating System
\item GCS - Ground Conrol Station
\item UAV - Unmanned Aerial Vehicle
\end{itemize}
%Provide a list of terms used in the document that warrant definition.  Consider 
%industry or domain specific terms and acronyms as well as system specific. 
