% !TEX root = sprints.tex
\\
\Large{\textbf{Team Overview}}
\\[-3mm]\noindent\makebox[\linewidth]{\rule{\textwidth}{0.4pt}}
\Large{\textbf{Name}}
\\ Expeditus 
\\[3mm]
\Large{\textbf{Members}}
\\ Jonathan Dixon, Dylan Geyer, Steven Huerta, Christopher Smith
\\[3mm]
\Large{\textbf{Project Title}}
\\ UAV Landing Pad
\\[3mm]
\Large{\textbf{Sponsor}}
\\ Dr. Larry Pyeatt, SDSMT MCS
\\[10mm]
\Large{\textbf{Sponsor Overview}}
\\[-3mm]\noindent\makebox[\linewidth]{\rule{\textwidth}{0.4pt}}
\Large{\textbf{Sponsor Description}}
\\ The Math and Computer Science Department of South Dakota School of Mines and Technology, in addition to providing ABET certified education to students, conducts software-side robotics research including autonomy, navigation, and computer vision. 
\\[3mm]
\Large{\textbf{Sponsor Problem}}
\\ The capability of UAVs to rapidly search a large area, especially one that is difficult to traverse by foot or vehicle, would be invaluable to operations such as search \& rescue. However, small UAVs have a very limited flight time. A system incorporating a UVG equipped with a landing pad that also serves as a charging station would allow the UAV to be delivered to areas of limited access. The UAV could then, being provided with waypoints by the user, autonomously take-off, and navigate through the waypoints. After moving through the waypoints, or when the UAV requires recharging, the UAV will return to the UVG and safely land in such a way that the charging unit can connect to the UAV.
\\[3mm]
\Large{\textbf{Sponsor Needs}}
\begin{itemize}
\item Ability to communicate waypoints to UAV.
\item UAV can autonomously take-off.
\item UAV can autonomously navigate through waypoints.
\item UAV can autonomously navigate back to landing pad.
\item UAV can autonomously land safely and with the correct orientation.
\end{itemize}
\vspace{6mm}
\Large{\textbf{Project Overview}}
\\[-3mm]\noindent\makebox[\linewidth]{\rule{\textwidth}{0.4pt}}
\Large{\textbf{Phase 1}}
This project is the continuation of previous years of research and development conducted by other Senior Design Teams. Previous development will be reviewed and implemented. The project will have the UAV capable of autonomous take-off and waypoint navigation. A simulation for landing tests will be completed.
\\[3mm]
\Large{\textbf{Phase 2}}
This project will complete the navigation to the landing pad, and orientation specific landing of the UAV. The UAV will be capable of autonomous take-off, waypoint navigation, and orientation-specific landing.
\\[10mm]
\Large{\textbf{Project Environment}}
\\[-3mm]\noindent\makebox[\linewidth]{\rule{\textwidth}{0.4pt}}
\Large{\textbf{Project Boundaries}}
\\[3mm]
\Large{\textbf{Project Context}}
\\[10mm]
\Large{\textbf{Deliverables}}
\\[-3mm]\noindent\makebox[\linewidth]{\rule{\textwidth}{0.4pt}}
\Large{\textbf{Phase 1}}
