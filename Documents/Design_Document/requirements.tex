% !TEX root = SystemTemplate.tex
\chapter{User Stories, Backlog and Requirements}
\section{Overview}
%The overview should take the form of an executive summary.  Give the reader a feel 
%for the purpose of the document, what is contained in the document, and an idea 
%of the purpose for the system or product. 

% The userstories 
%are provided by the stakeholders.  You will create he backlogs and the requirements, and document here.  
%This chapter should contain 
%details about each of the requirements and how the requirements are or will be 
%satisfied in the design and implementation of the system.

%Below:   list, describe, and define the requirements in this chapter.  
%There could be any number of sub-sections to help provide the necessary level of 
%detail. 
This section details the requirements(defined by the user stories) that set the goals, and subsequent tasks, of the project. This document will provide a more detailed view of the user stories. Additionally, this document will describe the client and developers more thoroughly.




\subsection{Scope}
This section includes the client information, user stories, and requirements. Proof of concept results will be added later.

%What scope does this document cover?  This document would contain stakeholder information, 
%initial user stories, requirements, proof of concept results, and various research 
%task results. 



\subsection{Purpose of the System}
The purpose of this project is to provide a prototype of implementing navigation and, more importantly, landing autonomy on a UAV

What is the purpose of the system or product? 


\section{ Stakeholder Information}


This section would provide the basic description of all of the stakeholders for 
the project.  Who has an interest in the successful and/or unsuccessful completion 
of this project? 


\subsection{Customer or End User (Product Owner)}
Who?  What role will they play in the project?  Will this person or group manage 
and prioritize the product backlog?  Who will they interact with on the team to 
drive product backlog priorities if not done directly? 

\subsection{Management or Instructor (Scrum Master)}
Who?  What role will they play in the project?  Will the Scrum Master drive the 
Sprint Meetings? 


\subsection{Investors}
There are no investors in this project.
%Are there any?  Who?  What role will they play? 


\subsection{Developers --Testers}
%Who?  Is there a defined project manager, developer, tester, designer, architect, 
%etc.? 


%\section{Business Need}
%Use this section to define what business need exist and how this software will 
%meet and/or exceed that business need.   

\section{Requirements and Design Constraints}
%Use this section to discuss what requirements exist that deal with meeting the 
%business need.  These requirements might equate to design constraints which can 
%take the form of system, network, and/or user constraints.  Examples:  Windows 
%Server only, iOS only, slow network constraints, or no offline, local storage capabilities. 


\subsection{System  Requirements}
%What are they?  How will they impact the potential design?  Are there alternatives? 


\subsection{Network Requirements}
%What are they? 


\subsection{Development Environment Requirements}
%What are they?  Is the system supposed to be cross-platform? 


\subsection{Project  Management Methodology}
The stakeholders might restrict how the project implementation will be managed. 
 There may be constraints on when design meetings will take place.  There might 
be restrictions on how often progress reports need to be provided and to whom. 
 
\begin{itemize}
\item What system will be used to keep track of the backlogs and sprint status?
\item Will all parties have access to the Sprint and Product Backlogs?
\item How many Sprints will encompass this particular project?
\item How long are the Sprint Cycles?
\item Are there restrictions on source control? 
\end{itemize}

\section{User Stories}
This section can really be seen as the guts of the document.  This section should 
be the result of discussions with the stakeholders with regard to the actual functional 
requirements of the software.  It is the user stories that will be used in the 
work breakdown structure to build tasks to fill the product backlog for implementation 
through the sprints.

This section should contain sub-sections to define and potentially provide a breakdown 
of larger user stories into smaller user stories. 



\subsection{User Story \#1}
User story \#1 discussed. 

\subsubsection{User Story \#1 Breakdown}
Does the first user story need some division into smaller, consumable parts by 
the reader?  This does not need to go to the level of actual task definition and 
may not be required. 

\subsection{User Story \#2} 

\subsubsection{User Story \#2 Breakdown}
User story \#2  .... 

\subsection{User Story \#3} 

\subsubsection{User Story \#3 Breakdown}
User story \#3  .... 


%\section{Research or Proof of Concept Results}
%This section is reserved for the discussion centered on any research that needed 
%to take place before full system design.  The research efforts may have led to 
%the need to actually provide a proof of concept for approval by the stakeholders. 
%The proof of concept might even go to the extent of a user interface design or 
%mockups. 


%\section{Supporting Material}
%This document might contain references or supporting material which should be documented 
%and discussed  either here if appropriate or more often in the appendices at the end.  This material may have been provided by the %stakeholders  
%or it may be material garnered from research tasks.

