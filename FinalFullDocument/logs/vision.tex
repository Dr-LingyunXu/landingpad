% !TEX root = SystemTemplate.tex
\section{Vision Log}

\begin{description}

\item [9/14/15]  Began looking into original code. \hfill{Jon Dixon}

\item [9/21/15]  Ran original code with landing pad target from the robotics lab. Code was able to detect the three blobs, and give what appeared to be an accurate distance reading in centimeters. \hfill{Jon Dixon}

\item [9/28/15]  Understood legacy vision code and was able to develop an idea of a plan for future sprints. \hfill{Jon Dixon}

\item [10/19/15]  Continued working with legacy vision code. Determined a plan of attack - find the angle of the target so that we can detect orientation. \hfill{Jon Dixon}

\item [10/26/15]  Continued with attempt to add orientation sensitivity, but not making any real progress. \hfill{Jon Dixon}

\item [11/9/15]  Developed a plan of attack to change from a three blob approach to blob detection to a four blob approach. This idea was prompted by the computer vision class, as we were taught that for full-blown homography it helps to have four known features to use for the math. \hfill{Jon Dixon}

\item [11/16/15]  Worked with Julian to develop a new landing target that would have four colored blobs instead of three. \hfill{Jon Dixon}

\item [11/23/15]  After playing with legacy code and attempting to make it work for the new four blob system, Julian and I determined that we had no idea how to actually go from a homography matrix to an orientation angle. \hfill{Jon Dixon}

\item[11/23/15] Began work and research with other vision tracking systems. One potential option is using hough transforms to search for circles in an image, since circles are going to be relatively uncommon in nature. Began research on QR/AR tag tracking libraries. \hfill{Jon Dixon}

\item [12/28/15]  Attempted to implement ar\_track\_alvar. Unable to get ros package working. Camera is working. Tag is being detected, but pose information is not being outputted. Changed sizes of AR tag, no change. Changed parameters in launch file, no change. \hfill{Steven Huerta}

\item [1/18/16] Got an ALVAR environment set up in Ubuntu 14.04 using an image stream from the webcam on the school laptop. ALVAR sample programs build and run successfully, but note that there is no ROS implementation in the raw ALVAR source.  \hfill{Dylan Geyer}

\item [1/25/16]  Attempting to get ar\_track\_alvar working. Using a webcam instead of firefly to see if results are different. They are not. Results are the same. Tag is being detected, including correct identification of tag type, but no pose information is being produced.  \hfill{Steven Huerta}


\item [2/1/16]  Non-ROS version of ALVAR is able to see AR Tags and estimate their pose. This information is stored as 3 translation variables(x,y,z) and 4 quaternian terms. Future implementations could bundle this data into a ROS Message but for now it is just printed to the terminal. \hfill{Dylan Geyer}

\item [2/15/16]  ar\_track\_alvar now working consistently and reliably. ROS package requires presence of camera calibration file. Absence of file will not produce errors, however no pose data will be outputted. Calibration MUST occur before ar\_track\_alvar package is implemented. \hfill{Steven Huerta}

\item [2/22/16]  Reproduced calibration files for both firefly and web cameras. Pose estimation is much more accurate. AR tag metrics must be adjusted if switching to a different sized tag. This must be done to get reliable and accurate pose estimates of the AR tag. \hfill{Steven Huerta}

\item [2/29/16]  Starting work on installing SVO project. SVO project is implementation of Semi-Direct FAST feature detection. SVO project package is designed to incorporate into the ROS environment. \hfill{Steven Huerta}

\item [3/21/16]  Needed to adapt camera calibration files to create necessary parameter files for SVO. SVO ATAN calibration method proved to be somewhat problematic. ATAN is recommended, but calibration application continues to fail during calibration. Files are now generated for both firefly and webcam. Both are using PINHOLE camera types. Hoping that this is not a deal breaker.  \hfill{Steven Huerta}

\item [3/28/16]  SVO continues to break regardless of camera being used. SVO can initially localize, but any transition or rotation of the camera results in a loss of features so that localization can no longer be calculated. Not sure where this error is coming from. If camera moves incredibly slowly and not too far, SVO will continue to report pose. We are very close.  \hfill{Steven Huerta}


\item [4/4/16] SVO continues to fail. Attempted to change some camera parameters (shutter speed, frame rate, and autofocus), and did not succeed with either firefly or web camera. Also attempted to change some of the SVO parameters in the launch file such as minimum features. Performance increased slightly, but not enough to overcome likely movement of UAV.  \hfill{Steven Huerta}
\end{description}