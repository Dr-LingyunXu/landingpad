% !TEX root = sprints.tex
\Large{\textbf{Team Overview}}
\\[-3mm]\noindent\makebox[\linewidth]{\rule{\textwidth}{0.4pt}}
\normalsize{\textbf{Name}}
\\Expeditus
\\[3mm]
\normalsize{\textbf{Members}}
\\ Jonathan Dixon, Dylan Geyer, Steven Huerta, Christopher Smith
\\[3mm]
\normalsize{\textbf{Project Title}}
\\ UAV Landing Pad
\\[3mm]
\normalsize{\textbf{Sponsor}}
\\ Dr. Larry Pyeatt, SDSMT MCS
\\[10mm]
\Large{\textbf{Sponsor Overview}}
\\[-3mm]\noindent\makebox[\linewidth]{\rule{\textwidth}{0.4pt}}
\normalsize{\textbf{Sponsor Description}}
\\ The Math and Computer Science Department of South Dakota School of Mines and Technology, in addition to providing ABET certified education to students, conducts software-side robotics research including autonomy, navigation, and computer vision. 
\\[3mm]
\normalsize{\textbf{Sponsor Problem}}
\\ The capability of UAVs to rapidly search a large area, especially one that is difficult to traverse by foot or vehicle, would be invaluable to operations such as search \& rescue. However, small UAVs have a very limited flight time. A system incorporating a UVG equipped with a landing pad that also serves as a charging station would allow the UAV to be delivered to areas of limited access. The UAV could then, being provided with waypoints by the user, autonomously take-off, and navigate through the waypoints. After moving through the waypoints, or when the UAV requires recharging, the UAV will return to the UVG and safely land in such a way that the charging unit can connect to the UAV.
\\[3mm]
\normalsize{\textbf{Sponsor Needs}}
\begin{itemize}
\item Ability to communicate waypoints to UAV.
\item UAV can autonomously take-off.
\item UAV can autonomously navigate through waypoints.
\item UAV can autonomously navigate back to landing pad.
\item UAV can autonomously land safely and with the correct orientation.
\end{itemize}
\vspace{6mm}
\Large{\textbf{Project Overview}}
\\[-3mm]\noindent\makebox[\linewidth]{\rule{\textwidth}{0.4pt}}
\normalsize{\textbf{Phase 1}}
First phase will focus on finalizing the autonomous take-off and waypoint navigation by the UAV. Previous development will be reviewed, implemented, and tested. Simulation environment will be created for the purpose of testing landing algorithms. 
\\[3mm]
\normalsize{\textbf{Phase 2}}
Second phase will focus on finalizing autonomous landing
\\[10mm]
\Large{\textbf{Project Environment}}
\\[-3mm]\noindent\makebox[\linewidth]{\rule{\textwidth}{0.4pt}}
\normalsize{\textbf{Project Boundaries}}
\begin{itemize}
\item Project is constrained to the UAV autonomy problems of take-off, navigation, and landing. 
\item Autonomous landing is constrained by fixed position landing platform with ideal operating conditions.
\item Autonomous take-off is constrained by taking flight from a fixed position platform, with ideal operating conditions.
\item Autonomous waypoint navigation is constrained by absence of obstacles, and operating with ideal operating conditions.
\end{itemize}
\vspace{3mm}
\normalsize{\textbf{Project Context}}
\begin{itemize}
\item Project will utilize stable ROS distribution
\item Project simulations will utilize Gazebo 6.+ \& ROS package Rviz
\item Project will be developed in Linux environment compliant with ROS \& Gazebo 
\end{itemize}
\vspace{6mm}
\Large{\textbf{Deliverables}}
\\[-3mm]\noindent\makebox[\linewidth]{\rule{\textwidth}{0.4pt}}
\normalsize{\textbf{Phase 1}}
\begin{itemize}
\item Requirements documentation
\item Overview documentation
\end{itemize}
\vspace{3mm}
\normalsize{\textbf{Phase 2}}
\begin{itemize}
\item Project software
\item Log
\item Refence manual (software documentation)
\item User documentation
\item System design documentation
\item Testing documentation
\item Deployment documentation
\end{itemize}
\vspace{6mm}
\Large{\textbf{Product Backlog}}
\\[-3mm]\noindent\makebox[\linewidth]{\rule{\textwidth}{0.4pt}}
\normalsize{\textbf{Phase 1}}
\begin{itemize}
\item \textbf{O-1}: As an owner, I want the UAV to autonomously take-off from the landing pad
\item \textbf{O-2}: As an owner, I want the UAV to autonomously navigate through a series of waypoints
\end{itemize}
\vspace{3mm}
\normalsize{\textbf{Phase 2}}
\begin{itemize}
\item \textbf{U-1}: As a user, I want to communicate the waypoints to the UAV
\item \textbf{O-3}: As an owner, I want the UAV to autonomously return to the location of the landing pad
\item \textbf{O-4}: As an owner, I want the UAV to autonomously land on the landing pad without damaging the craft
\item \textbf{O-5}: As an owner, I want the UAV to autonomously land on the landing pad with the correct orientation
\end{itemize}
\vspace{6mm}
\Large{\textbf{Sprint Report}}
\\[-3mm]\noindent\makebox[\linewidth]{\rule{\textwidth}{0.4pt}}
\normalsize{\textbf{Completed Tasks}}
\begin{itemize}
\item Install Ubuntu 14.04 or some other ROS Indigo/Jade distro compliant OS.
\item Setup Gazebo 6.+
\item Download Rviz package
\item Review previous iteration of project documentation
\item Inspect current quadrotor configuration
\item Identify parts needed for quadrotor
\end{itemize}
\vspace{3mm}
\normalsize{\textbf{Tasks Carried to Next Sprint}}
\begin{itemize}
\item Acquire parts needed for quadrotor
\end{itemize}
\vspace{6mm}